% platzhalten <\nameDesPlatzhalters> wird von matlab ersetzt

\newcommand{\itaBriefkopfBild}{<\itaBriefkopfBild>}
\newcommand{\datumDerMessung}{<\datumDerMessung>}
\newcommand{\nameDesPruefers}{<\nameDesPruefers>}
\newcommand{\temperaturInGradC}{<\temperaturInGradC>}
\newcommand{\luftfeuchtigkeit}{<\luftfeuchtigkeit>}
\newcommand{\nameDerProbe}{<\nameDerProbe>}
\newcommand{\anzahlDerMessungen}{<\anzahlDerMessungen>}
\newcommand{\samplesOderWiederholungen}{<\samplesOderWiederholungen>}
\newcommand{\dateinameGrafik}{<\dateinameGrafik>}
\newcommand{\smoothParameter}{<\smoothParameter>}
\newcommand{\beschreibungDesMaterials}{<\beschreibungDesMaterials>}

\newcommand{\terzWertA}{<\terzWertA>}  % 100
\newcommand{\terzWertB}{<\terzWertB>}  % 125
\newcommand{\terzWertC}{<\terzWertC>}  % 160
\newcommand{\terzWertD}{<\terzWertD>}  % 200
\newcommand{\terzWertE}{<\terzWertE>}  % 250
\newcommand{\terzWertF}{<\terzWertF>}  % 315
\newcommand{\terzWertG}{<\terzWertG>}  % 400
\newcommand{\terzWertH}{<\terzWertH>}  % 500
\newcommand{\terzWertI}{<\terzWertI>}  % 630
\newcommand{\terzWertJ}{<\terzWertJ>}  % 800


\newcommand{\terzWertK}{<\terzWertK>}  % 1000
\newcommand{\terzWertL}{<\terzWertL>}  % 1250
\newcommand{\terzWertM}{<\terzWertM>}  % 1600
\newcommand{\terzWertN}{<\terzWertN>}  % 2000
\newcommand{\terzWertO}{<\terzWertO>}  % 2500
\newcommand{\terzWertP}{<\terzWertP>}  % 3150
\newcommand{\terzWertQ}{<\terzWertQ>}  % 4000
\newcommand{\terzWertR}{<\terzWertR>}  % 5000
\newcommand{\terzWertS}{<\terzWertS>}  % 6300
\newcommand{\terzWertT}{<\terzWertT>}  % 8000




%\newcommand{\datumDerMessung}{22.11.2111}
%\newcommand{\nameDesPruefers}{Martin Guski}
%\newcommand{\temperaturInGradC}{21,5}
%\newcommand{\luftfeuchtigkeit}{52,3}
%\newcommand{\nameDerProbe}{NMC A}
%\newcommand{\anzahlDerMessungen}{11}
%\newcommand{\samplesOderWiederholungen}{Samples}
%\newcommand{\dateinameGrafik}{NMC-A.png}
%\newcommand{\smoothParameter}{12}
%\newcommand{\beschreibungDesMaterials}{	Anthrazitfarbiger, teils geschlossen-poriger Zellschaum. Dicke ca. 52 mm
%	Passgenaue Platzierung der Samples im Probenhalter mit dichtem Abschluss zur Wandung. 
%	Einbau der Proben ohne Abstand vor dem schallharten Rohrabschluss.}
%
%\newcommand{\terzWertA}{0.01}  % 100
%\newcommand{\terzWertB}{0.02}  % 125
%\newcommand{\terzWertC}{0.03}  % 160
%\newcommand{\terzWertD}{0.04}  % 200
%\newcommand{\terzWertE}{0.05}  % 250
%\newcommand{\terzWertF}{0.06}  % 315
%\newcommand{\terzWertG}{0.07}  % 400
%\newcommand{\terzWertH}{0.08}  % 500
%\newcommand{\terzWertI}{0.09}  % 630
%\newcommand{\terzWertJ}{0.10}  % 800
%
%
%\newcommand{\terzWertK}{0.11}  % 1000
%\newcommand{\terzWertL}{0.12}  % 1250
%\newcommand{\terzWertM}{0.13}  % 1600
%\newcommand{\terzWertN}{0.14}  % 2000
%\newcommand{\terzWertO}{0.15}  % 2500
%\newcommand{\terzWertP}{0.16}  % 3150
%\newcommand{\terzWertQ}{0.17}  % 4000
%\newcommand{\terzWertR}{0.18}  % 5000
%\newcommand{\terzWertS}{0.19}  % 6300
%\newcommand{\terzWertT}{0.20}  % 8000









\documentclass[11pt,a4paper, twoside, final]{scrartcl}
\usepackage{graphicx}
\usepackage[latin1]{inputenc}
\usepackage{lscape}

\pagestyle{plain}

\setlength{\textwidth}{17.5cm}
\setlength{\oddsidemargin}{-1cm}
\setlength{\evensidemargin}{1cm}
\setlength{\topmargin}{-2cm}
\setlength{\headheight}{0cm}
\setlength{\headsep}{0cm}
\setlength{\footskip}{0cm}
\setlength{\textheight}{27.5cm}

\pagestyle{empty}



\begin{document}

\parindent 0pt 

 %\includegraphics[width=1.0\textwidth]{\itaBriefkopfBild}

\begin{center}
	 \LARGE
		Measurement report: Determination of sound absorption coefficient in the impedance tubes according to EN ISO 10534 part 2
\end{center}

\begin{tabular}{|p{0.43\textwidth}p{0.51\textwidth}|}\hline
	\multicolumn{2}{|p{0.99\textwidth}|}{
	\underline{Description of the test facility:} \newline
Impedance  tube made of aluminum with a 2 inch bore diameter.  Specimen holder with an adjustable tubing end of a massive, sealed aluminum piston. Measurement tube with four microphone bores, the distance from the first microphone position to the second, third and fourth are 17~mm, 110~mm and 510~mm, respectively.  Probe microphone picking up the sound field at the center of the tube. 
Evaluable frequency range: approx. 100 Hz - 8 kHz by suppressing the first tube modes.  Broadband excitation by sine sweeps. Evaluation according to ISO10534-2 with the help of a FFT measurement technique developed at ITA.}\\ 
	&\\
		Test date: & \datumDerMessung \\
		 Test Engineer: & \nameDesPruefers\\ 
	\multicolumn{2}{|l|}{
		Meteorological conditions:   \hspace{3.175cm} Temperature: \temperaturInGradC ~�C  \qquad  \qquad Humidity:  \luftfeuchtigkeit ~\% }\\ \hline 
	
	\multicolumn{2}{c}{ }\\ \hline


	\multicolumn{2}{|p{0.99\textwidth}|}{	\underline{Description of the materials to be examined and the testing arrangement:}\newline
	\beschreibungDesMaterials }\\
	 & \\
	Designation of the examined sample: & \nameDerProbe \\ 
	Number of  \samplesOderWiederholungen : & \anzahlDerMessungen \\ \hline
	
\end{tabular}


\begin{figure}[htbp]
\centering
		\includegraphics[width=1\textwidth]{\dateinameGrafik}
	\caption{Mean value of the frequency-dependent absorption coefficient (solid curve) and standard deviation of measurement result (dotted curves). Calculation based on  the measurements of \anzahlDerMessungen \ \samplesOderWiederholungen \ . Curves smoothed by applying moving average: smoothing parameter \smoothParameter \ octave, Gaussian window. }
	\label{fig:ZweiProben_gemittelt}
\end{figure}

\small
\centering
\begin{tabular}{|l||c|c|c|c|c|c|c|c|c|c|c|c|c|c|c|c|c|c|c|c|} \hline
 $f / [Hz]$  &  125  &  160  &  200  &  250  &  315  &  400  &  500  &  630  &  800  \\ 
 $\alpha$     &  \terzWertB &  \terzWertC &  \terzWertD &  \terzWertE &  \terzWertF &  \terzWertG &  \terzWertH &  \terzWertI &  \terzWertJ  \\ \hline \hline 
 $f / [Hz]$ &  1000 &  1250 &  1600 &  2000 &  2500 &  3150 &  4000 &  5000 &  6300   \\ 
 $\alpha$   &  \terzWertK &  \terzWertL &  \terzWertM &  \terzWertN &  \terzWertO &  \terzWertP &  \terzWertQ &  \terzWertR &  \terzWertS   \\ \hline
\end{tabular}
\\

%\vspace{2cm}


\end{document}